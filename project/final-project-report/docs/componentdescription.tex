\subsection{Komponentbeskrivelse}\label{sec:blockdescription}
Her beskrives komponenterne som blev vist på blokdiagrammet på side~\pageref{fig:blockdiagram}. 

\subsubsection{Bruger}
Brugeren af systemet vil være interesseret i at kunne indhente forskellige målinger fra et system, som kan være placeret på en vilkårlig position, så længe der er mobildækning.

\subsubsection{MC35}
En Siemens MC35 GSM/GPRS Modem (vist på figur~\ref{fig:devicegsm}) er i projektet anvendt til at sende og modtage sms beskeder, som gør kommunikation mellem bruger og system mulig. Under udvikling har to forskellige enheder været brugt. Med den første enhed var der en del inkonsistens i opførsel under udvikling, men da denne enheds blev udskiftet stoppede problemerne og debugging af problemer blev meget lettere.
Der interfaces mellem ATmega32 og MC35 ved hjælp af en UART forbindelse med en baudrate\footnote{Symboler (signalpulser) per sek.} på 38400.
MC35 har autobauding slået til som fabriks indstilling, hvilket gør det muligt at kommunikere med modemet en række forskellige understøttede hastigheder.
Med autobauding er der dog nogle særlige restriktioner og forudsætninger der skal overholdes:

\begin{itemize}
	\item 3 til 5 sek. delay før den første char i en kommando sendes.
	\item 8 data bits operation, ingen paritet og 1-bit stop.
\end{itemize}}

Ved initiering af ATmega32s USART, konfigureres registret UCSRB\footnote{USART Control and Status Register B} 
som kan ses på figur~\ref{fig:regucsrb}. Da der i dette projekt benyttes et interrupt på RXC flaget initieres RXCIE også.
På figur~\ref{fig:ucszbitsettings} ses data bit konfigurationerne. Da UCSZ0=1, UCSZ1=1 og UCSZ2=0 er default værdier på 
ATmega32 skal de ikke ændres på.

\begin{figure}[h]
	\centering
	\includegraphics[width=0.7\linewidth]{figs/ucsz_bitsettings.jpg}
	\caption{UCSZ bit settings på ATmega32.}
	\label{fig:ucszbitsettings}
\end{figure}

Se kodeudsnit~\ref{code:initucsrb} for initiering af USARTen. 

Som nævnt sættes RXC interrupt flaget når der modtages data fra MC35.
Der er skrevet en Interrupt Service Routine der gemmer den modtagne data i et char* array, hvorefter programmet på ATmega32 behandler dataen
og reagerer på eventuelle kommandoer. Hver gang en dataen er behandlet, slettes indholdet af arrayet og er derfor klargjort til næste 
kommando. 

\begin{lstlisting}[caption=Initiering af UCSRB,label=code:initucsrb] 
UCSRB = ((1<<TXEN)|(1<<RXEN) | (1<<RXCIE));
\end{lstlisting}

\begin{figure}[h]
	\centering
	\includegraphics[width=0.7\linewidth]{figs/avr_register_ucsrb.jpg}
	\caption{ATmega32 UCSRB register.}
	\label{fig:regucsrb}
\end{figure}


\begin{figure}[h]
	\centering
	\includegraphics[width=0.7\linewidth]{figs/device_gsm.jpg}
	\caption{Siemens MC35 GSM/GPRS modem.}
	\label{fig:devicegsm}
\end{figure}

\subsubsection{ATmega32}
ATmega32 er brugt i forbindelse med STK500 boardet, som hovedkomponent i udviklingen. Denne er vist på figur~\ref{fig:deviceatmege32}. 
Boardet interfacer med GSM modemet via UART og fortolker de modtagne kommandoer før der kan anmodes om de ønskede målinger fra BMP modulet.

\begin{figure}[h]
	\centering
	\includegraphics[width=0.7\linewidth]{figs/device_atmega32.jpg}
	\caption{STK500 board opsætning med ATmega32.}
	\label{fig:deviceatmege32}
\end{figure}

\subsubsection{BMP}
BMP08 modulet er brugt til at foretage målingerne og kommunikerer med ATmega32 over I2C. Den er i stand til at måle temperatur, lufttryk og altitude. 
Disse tre forskellige målinger kan så anmodes af ATmega32 og senders ud til brugeren. Vist på figur~\ref{fig:devicebmp}.

\begin{figure}[h]
	\centering
	\includegraphics[width=0.7\linewidth]{figs/device_bmp.jpg}
	\caption{BMP085 digital trykføler.}
	\label{fig:devicebmp}
\end{figure}
